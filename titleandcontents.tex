\pagestyle{empty}

\title{\LARGE{\thesistitle} \\[3cm]}

\author{\Large{\authorname} \\ \Large{\collegename} \\[3cm]}

\date{\DPhiltext \\[1cm] \DPhildate}

\maketitle

\cleardoublepage

\cleardoublepage

\begin{abstract}
\abstext
\end{abstract}

%\cleardoublepage
%\pagestyle{plain}
%\pagenumbering{roman}
%\raggedbottom
%
%\vspace*{10cm}
%
%Dedication.

\cleardoublepage

\section*{Acknowledgements}

I count myself very lucky to have been able to be part of two exceptional groups at the University of Oxford and at CERN. Of course, in those groups I owe the most to my supervisors, Glenn Christian, at Oxford, and Piotr Skowronski, at CERN. 
Your unbreakable resolve and experience has unquestionably driven the PFF system far beyond what I thought was possible, and I have learnt so much from both of you, having started the DPhil with almost no experience of accelerator physics or feedback/feedforward systems.
Your patience and guidance in reading this thesis alone prove the incredible support you have both given me.

In the FONT group at Oxford I must also give my sincerest gratitude to Philip Burrows, not least for the opportunity to join the group in the first place. He has created a wonderful group to be a part of, and I have enjoyed countless discussions with him, Glenn Christian, and all the other group members, current and alumni, whether it be in meetings over the phone, in person or at lunch or dinner. In particular I would like to thank Colin Perry, for his efforts with the amplifier, and Neven Blaskovic, for welcoming me to the group with open arms.

Thanks as well to the whole CTF3 team at CERN, headed by Roberto Corsini, who himself has provided invaluable direction and support in pushing forward the PFF system. A special mention to the core team of operators during my time here: Frank Tecker, Steffen Doebert, Davide Gamba, Tobias Persson, Luis Navarro, Lukas Malina, and, of course, Piotr Skowronski. Without your incredible efforts to keep CTF3 at its best, against all odds and late in to the night at times, none of the work in this thesis would have been possible. On a more personal note, a particularly warm mention to those of you with whom I shared unforgettable experiences, and in some cases rooms, when travelling to various conferences and workshops, especially in Belgrade and Richmond. %And a fond farewell to CTF3 itself, which is in its last months of operation in its current capacity, having provided a unique opportunity to gain hands-on experience and training with a real accelerator for countless people.

The key challenge of the PFF system is the hardware, and although I do not know all of you personally I am incredibly grateful for everyone involved in designing and constructing the necessary components: Colin Perry for the amplifier and FONT5a boards; Glenn Christian for the FONT5a board firmware; Douglas Bett for the FONT5a board DAQ; Fabio Marcellini and Andrea Ghigo for the phase monitors and kickers; Alexandra Andersson, Luca Timeo and Stephane Rey for the phase monitor electronics; and Piotr Skowronski for the initial conception and integration of the PFF system at CTF3. 

I would also like to thank all my friends, in particular Espen Bowen and Kara Lynch, who have been here for the whole time I've been at CERN during my DPhil. And to Magdalena, Monika and Christiane for their support in and around the office whilst I was writing the thesis. To everyone involved with the Le Box and BBQ Afterwards football teams over the years - thank you for putting up with my questionable talent.

Last but certainly not least, thank you to my family, Chris, Elizabeth and Kate Roberts, for their unwavering, unconditional cheerleading and support, especially during the weeks I spent at home whilst writing the thesis.

There are so many other people that have played a part in me completing this thesis over the years, both professionally and socially. In a last ditch attempt to keep the thesis brief I cannot name you all, but thanks to all of you as well.

\cleardoublepage

\tableofcontents
%\listoffigures
%\listoftables


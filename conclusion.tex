\newchapter{conclusion}{Conclusions}

\newsection{summary}{Summary}

CLIC is a proposal for a future linear electron--positron collider in which 12~GHz, normal conducting accelerating cavities with an accelerating gradient of 100~MV/m are used to achieve collision energies of up to 3~TeV. The RF power for these cavities is extracted from a second, high intensity, drive beam. This two beam acceleration concept is a unique feature of CLIC and the generation of and power extraction from the drive beam presents many challenges. In particular, as the energy of the main beam is determined by the properties of the drive beam there are strict tolerances on the drive beam stability. One such constraint is on the phase stability, which must be \(0.2^\circ\)~at~12~GHz or better to limit luminosity loss at the collision point resulting from energy errors to below 1\%. 

As the expected drive beam phase stability is \(2^\circ\)~at~12~GHz CLIC requires the use of a ``phase feedforward'' (PFF) system, which will improve the phase stablity by an order of magnitude. In the PFF system kickers are used to deflect the drive beam on to longer or shorter paths in a chicane. Based on the phase measured in a monitor upstream of the chicane the kicker voltage is varied so that the bunches arriving early at the monitor are deflected on to longer paths, and vice versa, so that the phase is on reference at the exit of the chicane. By placing the monitor before a turnaround in the beam line the correction signals can travel a shorter distance between the monitor and the kickers than the beam, thus the correction can be applied to exactly the same beam pulse that was initially measured. This thesis has documented the design, commissioning and operation of a prototype PFF system at the CLIC test facility CTF3.

Changes were made to the TL2 line at CTF3 in order to accommodate the PFF kickers. In Chapter~\ref{c:tl2Optics} new optics were created for TL2 to take in to account these changes and to create the desired phase shifting behaviour in the chicane. Optics measurements were first taken to identify and remove errors in the MADX model of the line. Adjusting the focusing strength of quadrupoles and dipoles along the line reduced the discrepancy between the measurements and the model by an order of magnitude. With the corrections in place the new PFF optics were matched to give the largest possible value of the transfer matrix coefficient \(R_{52}\), which defines the correction range, whilst maintaining constraints on the dispersion, twiss parameters and \(R_{56}\). An optics was found with an \(R_{52}\) value of 0.74~m and dispersion below 1~m, but a non-zero \(R_{56}\) value of -0.18~m had to be accepted.

With non-zero \(R_{56}\) in TL2 the downstream phase jitter (following the correction chicane) is expected to contain an energy dependent component that is not present in the upstream phase (used as the PFF input). This reduces the correlation between the upstream and downstream phase, which must be 97\% to reduce an initial phase jitter of \(0.8^\circ\) to the targeted \(0.2^\circ\) with the PFF system. In Chapter~\ref{c:phasePropagation} the first measurements of the upstream-downstream phase correlation were below 40\%. To compensate for the negative \(R_{56}\) in TL2 new optics with positive \(R_{56}\) values have been used in the transfer line TL1 so that the overall \(R_{56}\) between the monitors is zero. By fine-tuning the \(R_{56}\) in TL1 upstream-downstream phase correlations of up to 93\% have been achieved. However, a strong second order (\(T_{566}\)) effect has also been identified, which causes apparent optimal \(R_{56}\) value to vary with energy. Relative beam energy jitter and variations along the pulse must therefore be kept below the \(10^{-3}\) level to be able to maintain high correlations. 

Another requirement to be able to theoretically achieve \(0.2^\circ\) corrected phase jitter with the PFF system is for the phase monitor resolution to be better than \(0.14^\circ\). Achieving this target required extensive measurements of the phase monitor electronics, as well as several changes as described in Chapter~\ref{c:phaseMons}. Tests of the electronics with a signal generator determined that the power dependent diode outputs saturated at a much lower power level than the phase dependent mixer outputs. The diodes were therefore excluded from the phase reconstruction process to be able to use higher input powers and increase signal to noise on the mixer outputs. Digital phase shifters initially used in the reference phase (LO) were found to be introducing noise to the system and were limiting the resolution to \(0.4^\circ\). Replacing these with mechanical phase shifters then yeilded an immediate resolution improvement. This combined with several other improvements, such as reducing digitiser noise, finally yielded \(0.13^\circ\) resolution, better than required.

Chapter~\ref{c:commissioning} dealt with the setup and commissioning of the remaining pieces of hardware for the PFF system -- the PFF controller (FONT5a board) and the kicker amplifiers. On the FONT5a board droop in the response of the ADCs was removed with the implementation of IIR filters, the conversion factor between the 14-bit gain set in the firmware and the true applied gain was derived and the correction output timing was setup. An optimal output delay of around 20~ns was determined in order to synchronise the voltage sent to the kickers with the arrival of the beam. The output voltage versus input voltage of the amplifier was characteried, with the output found to be linear in the range between \(\pm1.2\)~V input and a voltage gain of between \(409\pm3\) and \(453\pm3\) in this range. Variations in the output along the pulse were identified, which may cause imperfections in the phase correction, but these are small and at a level similar to or below the phase monitor resolution. When applied to the kickers the maximum amplifier output voltage yields a phase correction range of \(5.5\pm0.3^\circ\).

Finally, with the hardware and optics setup optimised Chapter~\ref{c:feedforward} presents the best results achieved with the PFF system to date. An initial mean phase jitter of \(0.74\pm0.06^\circ\) has been corrected to \(0.28\pm0.02^\circ\) using the PFF prototype, thereby achieving close to the CLIC target of \(0.2^\circ\) phase stability. The achieved jitter is in agreement with the theoretical prediction of \(0.27\pm0.02^\circ\) given the beam conditions at that time. The current limitations of the system were also discussed. In a longer dataset corresponding to several hours of operation the jitter was reduced from \(1.40\pm0.03^\circ\) to \(0.72\pm0.01\). The larger corrected jitter on these timescales is dominated by drifts in beam conditions, leading to a theoretical best possible corrected jitter of \(0.61\pm0.01^\circ\). The difference between the achieved and theoretical jitter could be reduced by automatically adjusting the PFF setup, gain and offset (zero point), based on the current beam conditions. However, to be able to achieve much lower corrected phase jitters on longer timescales new feedbacks are being implemented at CTF3 to improve the reproducibility of the beam conditions. 

\newsection{futureWork}{Future Work}

Optics -- attempt to match with zero R56/T566. Larger R52?

Phase Mons -- diodes/electronics design. Noise sources in LO -- phase shifters, multipliers, amplifiers, klystron noise? Position dependence (include as PFF input?).

Phase propagation -- basically same as optics constraints? Further fine tuning R56. Better energy stability needed.

Commissioning -- effects on amplifier (different outputs, droop etc.) could be taken in to account. Orbit closure. Timing difference between methods.

PFF goals -- 0.2 degrees jitter, factor 10 reduction. Combined beam?

PFF setup -- was not designed with quickly varying beam conditions in mind. Automatic updating of offset (combined with slow correction), gain. Zero point.

Recommendations for CLIC...
Optics control (chicane design, R56 and T566, or use energy as secondary input)
Phase monitor electronics design/performance







\newchapter{conclusion}{Conclusions}

\newsection{summary}{Summary}

CLIC is a proposal for a future linear electron--positron collider in which 12~GHz, normal conducting accelerating cavities with an accelerating gradient of 100~MV/m are used to achieve collision energies of up to 3~TeV. The RF power for these cavities is extracted from a second, high intensity, drive beam. This two beam acceleration concept is a unique feature of CLIC and the generation of and power extraction from the drive beam presents many challenges. In particular, as the energy of the main beam is determined by the properties of the drive beam there are strict tolerances on the drive beam stability. One such constraint is on the phase stability, which must be \(0.2^\circ\)~at~12~GHz or better to limit luminosity loss at the collision point resulting from energy errors to below 1\%. 

As the expected drive beam phase stability is \(2^\circ\)~at~12~GHz CLIC requires the use of a ``phase feedforward'' (PFF) system, which will improve the phase stablity by an order of magnitude. In the PFF system kickers are used to deflect the drive beam on to longer or shorter paths in a chicane. Based on the phase measured in a monitor upstream of the chicane the kicker voltage is varied so that the bunches arriving early at the monitor are deflected on to longer paths, and vice versa, so that the phase is on reference at the exit of the chicane. By placing the monitor before a turnaround in the beam line the correction signals can travel a shorter distance between the monitor and the kickers than the beam, thus the correction can be applied to exactly the same beam pulse that was initially measured. This thesis has documented the design, commissioning and operation of a prototype PFF system at the CLIC test facility CTF3.

Changes were made to the TL2 line at CTF3 in order to accommodate the PFF kickers. In Chapter~\ref{c:tl2Optics} new optics were developed for TL2, in order to take these changes in to account and to create the desired phase shifting behaviour in the chicane. Optics measurements were first taken to identify and remove errors in the MADX model of the line. Adjusting the focusing strength of quadrupoles and dipoles along the line reduced the discrepancy between the measurements and the model by an order of magnitude. With the corrections in place the new PFF optics were matched to give the largest possible value of the transfer matrix coefficient \(R_{52}\), which defines the correction range, whilst maintaining constraints on the dispersion, twiss parameters and \(R_{56}\). An optics was found with an \(R_{52}\) value of 0.74~m and dispersion below 1~m, but a non-zero \(R_{56}\) value of -0.18~m had to be accepted.

With non-zero \(R_{56}\) in TL2 the downstream phase jitter (following the correction chicane) is expected to contain an energy dependent component that is not present in the upstream phase (used as the PFF input). This reduces the correlation between the upstream and downstream phase, which must be 97\% to reduce an initial phase jitter of \(0.8^\circ\) to the targeted \(0.2^\circ\) with the PFF system. In Chapter~\ref{c:phasePropagation} the first measurements of the upstream-downstream phase correlation were below 40\%. To compensate for the negative \(R_{56}\) in TL2 new optics with positive \(R_{56}\) values have been used in the transfer line TL1 so that the overall \(R_{56}\) between the monitors is zero. By fine-tuning the \(R_{56}\) in TL1 upstream-downstream phase correlations of up to 93\% have been achieved. However, a strong second order (\(T_{566}\)) effect has also been identified, which causes apparent optimal \(R_{56}\) value to vary with energy. Relative beam energy jitter and variations along the pulse must therefore be kept below the \(10^{-3}\) level to be able to maintain high correlations. 

Another requirement to be able to theoretically achieve \(0.2^\circ\) corrected phase jitter with the PFF system is for the phase monitor resolution to be better than \(0.14^\circ\). Achieving this target required extensive measurements of the phase monitor electronics, as well as several changes as described in Chapter~\ref{c:phaseMons}. Tests of the electronics with a signal generator determined that the power dependent diode outputs saturated at a much lower power level than the phase dependent mixer outputs. The diodes were therefore excluded from the phase reconstruction process to be able to use higher input powers and increase signal to noise on the mixer outputs. Digital phase shifters initially used in the reference phase (LO) were found to be introducing noise to the system and were limiting the resolution to \(0.4^\circ\). Replacing these with mechanical phase shifters then yeilded an immediate resolution improvement. This combined with several other improvements, such as reducing digitiser noise, finally yielded \(0.13^\circ\) resolution, better than required.

Chapter~\ref{c:commissioning} dealt with the setup and commissioning of the remaining pieces of hardware for the PFF system -- the PFF controller (FONT5a board) and the kicker amplifiers. On the FONT5a board droop in the response of the ADCs was removed with the implementation of IIR filters, the conversion factor between the 14-bit gain set in the firmware and the true applied gain was derived and the correction output timing was setup. An optimal output delay of around 20~ns was determined in order to synchronise the voltage sent to the kickers with the arrival of the beam. The output voltage versus input voltage of the amplifier was characteried, with the output found to be linear in the range between \(\pm1.2\)~V input and a voltage gain of between \(409\pm3\) and \(453\pm3\) in this range, depending on the amplifier channel. Variations in the output along the pulse were identified, which may cause imperfections in the phase correction, but these are small and at a level similar to or below the phase monitor resolution. When applied to the kickers the maximum amplifier output voltage yields a phase correction range of \(5.5\pm0.3^\circ\).

Finally, with the hardware and optics setup optimised Chapter~\ref{c:feedforward} presents the best results achieved with the PFF system to date. An initial mean phase jitter of \(0.74\pm0.06^\circ\) has been corrected to \(0.28\pm0.02^\circ\) using the PFF prototype, thereby achieving close to the CLIC target of \(0.2^\circ\) phase stability. The achieved jitter is in agreement with the theoretical prediction of \(0.27\pm0.02^\circ\) given the beam conditions at that time. The current limitations of the system were also discussed. In a longer dataset corresponding to several hours of operation the jitter was reduced from \(1.40\pm0.03^\circ\) to \(0.72\pm0.01\). The larger corrected jitter on these timescales is dominated by drifts in beam conditions, leading to a theoretical best possible corrected jitter of \(0.61\pm0.01^\circ\). The difference between the achieved and theoretical jitter could be reduced by automatically adjusting the PFF setup based on the current beam conditions. However, to be able to achieve much lower corrected phase jitters on longer timescales new feedbacks are being implemented at CTF3 to improve the reproducibility of the beam conditions. 

\newsection{futureWork}{Future Work}

In terms of proving the feasibility of the PFF concept the prototype at CTF3 has fulfilled its goal.  The most important remaining task in the context of CLIC is to use the experience gained with the prototype to make recommendations for the design of the CLIC PFF system. Probably the most critical area here is the strict optics constraints needed to achieve 99.5\% upstream--downstream phase correlation, as required to reduce the initial jitter by an order of magnitude. The optics must be such that the chicane and turnaround introduce no additional first order (\(R_{56} = 0\)) or higher order (\(T_{566} = 0\)) energy dependent phase jitter. Using an energy measurement as a secondary input to the PFF system could also be considered, which would ensure that any uncorrelated energy component in the downstream phase jitter can be removed. There are also areas where hardware development is needed. Conceptual design of the higher power, 500~kW, amplifiers needed for CLIC has already begun \cite{colinCLIC16}. Modifications to the phase monitor electronics will also be necessary, in particular to address the issues at CTF3 that prevented the use of the diodes to create a power independent measurement. The resolution of the phase measurement should ideally be \(0.1^\circ\) so that the correction is not resolution limited very close to the \(0.2^\circ\) target. Based on the sensitivity of these devices to small changes in the environment at CTF3 it would also be useful to have the possibility of an in-situ resolution measurement, in the ideal case by installing pairs of monitors together in the beam line.

In each chapter areas where improvements could be made to the prototype at CTF3 have been identified, and future tests will attempt to address some of these issuse. Although it will be difficult to achieve significantly better than the \(0.27^\circ\) jitter result it is targeted to demonstrate close to this stability on much longer timescales, closer to the timescale of an hour rather than minutes. To date a reduction in downstream jitter by around a factor 2--3 has been achieved, whereas CLIC will have a larger initial jitter and require a factor 10 reduction. Tests at CTF3 will also be performed in which the initial phase jitter is artificially increased closer to the expected CLIC conditions, which will yield higher correlations and allow a larger relative reduction in phase jitter with the PFF system. As mentioned previously, the largest source of improvement will come from work to increase the stability of CTF3, in particular feedbacks to reduce energy jitter and drifts, with an additional small improvement possible by developing automatic procedures to determine the optimal PFF gain and other settings.

There are several new PFF setups that could also be attempted, each of which would require new logic and firmware for the correction. One example is attempting to run the PFF system with combined beam -- measuring the uncombined upstream phase and correcting the downstream, combined beam pulse. If shown to be feasible CLIC could then consider using one PFF system per drive beam (2 systems total), rather than one per drive beam decelerator sector (48 systems total). In addition, there are several small hardware effects that in principle could be removed with adjustments to the PFF algorithm. The simplest example of this would be using a non-linear DAC ouput from the FONT5a board to compensate for droop and other small variations in the amplifier response along the pulse. A first iteration of this could be achieved by using IIR filters on the DACs, similar to what has already been implemented for the ADCs. The current PFF firmware also provides the functionality to be able to use a secondary correction input \cite{glennPriv}. This was implemented to allow either the position dependence of the phase measurement (Section~\ref{s:monPosition}) or the energy dependence of the downstream phase to be taken in to account in the correction. However, the position dependence only makes a small contribution to the overall phase monitor resolution and the energy dependence of the downstream phase has been greatly reduced with the \(R_{56}\) optimisations. A two input correction is therefore unlikely to provide a significant improvement, but would still be useful as a proof of principle.

In terms of the optics and phase propagation, an attempt to create optics with lower \(T_{566}\) between the upstream and downstream phase monitors could be worthwhile, requiring the use of sextupoles. This would loosen the constraints on the energy stability required to achieve high upstream--downstream phase correlation. Alternatively, it may also be interesting to attempt to create an optics with larger \(R_{52}\) in the TL2 chicane (to give a larger correction range) whilst tolerating larger \(R_{56}\) values. The larger \(R_{56}\) value could then by compensated by varying the TL1 optics, in the same way as before. However, it is likely that larger dispersion in TL2 would also have to be accepted to be able to achieve this. Demonstrating that the prototype can be operated whilst maintaining orbit closure would also be a useful demonstration for CLIC, requiring further work to achieve beam transport with a completely nominal optics in TL2 and also using a different correction gain for each kicker to take in to account differences between the amplifier outputs.


%Phase Mons -- Noise sources in LO.

%Commissioning -- Timing difference between methods.






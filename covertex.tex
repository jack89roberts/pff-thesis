% Place the title and abstract in macros so the same version can be included
% in both the main text and the separate abstract required by the Bodleian.

\newcommand{\thesistitle}{Development of a Beam-based Phase Feedforward Demonstration at the CLIC Test Facility (CTF3).}

\newcommand{\authorname}{Jack Roberts}

\newcommand{\collegename}{New College, Oxford}

\newcommand{\DPhiltext}{Thesis submitted in fulfilment of the requirements for the degree of Doctor of Philosophy at the University of Oxford}

\newcommand{\DPhildate}{Trinity Term, 2016}

\newcommand{\abstext}{
The Compact Linear Collider (CLIC) is a proposal for a future linear electron--positron
collider that could achieve collision energies of up to 3~TeV. In the CLIC concept the main high energy beam is accelerated using RF power extracted from a high intensity drive beam, achieving an accelerating gradient of 100~MV/m. This scheme places strict tolerances on the drive beam phase stability, which must be better than \(0.2^\circ\) at 12~GHz. To achieve the required phase stability CLIC proposes a high bandwidth (\({>}17.5\)~MHz), low latency drive beam ``phase feedforward'' (PFF) system. In this system electromagnetic kickers, powered by 500~kW amplifiers, are installed in a chicane and used to correct the phase by deflecting the beam on to longer or shorter trajectories. A prototype PFF system has been installed at the CLIC Test Facility CTF3, the design, operation and commissioning of which is the focus of this work.

Two kickers have been installed in a pre-existing chicane at CTF3 for the prototype. New optics have been created for the line to take these changes in to account, incorporating new constraints to obtain the desired phase shifting behaviour. Three new phase monitors have also been installed, one for the PFF input and two to verify the system performance.  To achieve a \(0.2^\circ\) correction a phase monitor resolution below \(0.14^\circ\) is required. A point by point resolution sampled at 192~MHz below \(0.13^\circ\) has been achieved after a series of measurements and improvements to the phase monitor electronics. 

The performance of the PFF system depends on the correlation between the PFF input phase and the beam phase after the correction chicane. Preliminary measurements found only 40\% correlation. The source of the low correlation was determined to be energy dependent phase jitter, which has been removed by further optics adjustments. A final correlation of 93\% was achieved, improving the theoretical reduction in jitter using the PFF system from a factor 1.1 to a factor 2.7.

The performance and commissioning of the kicker amplifiers and PFF controller are also discussed. Beam based measurements are used to determine the optimal correction timing. With a maximum output of around 650~V the amplifiers facilitate a correction range of \(\pm5.5\pm0.3^\circ\). Finally, results from operation of the complete system are presented. A mean phase jitter of \(0.28\pm0.02^\circ\) is achieved, in agreement with the theoretical prediction of \(0.27\pm0.02^\circ\) for an optimal system in the given beam conditions. Current limitations of and possible future improvements to the PFF setup are also discussed.

}
